\documentclass[12pt]{article}
\newcommand{\helperdir}{/home/jacob/Documents/Latex-Documents/Helpers}
\newcommand{\codedir}{../Code}
\newcommand{\figuredir}{../Figures}
\newcommand{\datadir}{../Data}

\usepackage{amsmath}  %for begin align
\usepackage{footnote} %for \footnote{}

\usepackage[hidelinks]{hyperref}	%for \url{}. hidelinks to remove green border around links
\usepackage[shortlabels]{enumitem}	%to be  able to change the enum characters (e.g. \begin{enumerate}[i)]

\input{\helperdir/all.tex}
\pagestyle{plain} %for fancy footer to work correctly

\usepackage{tipa} %for \textsci


\begin{document}
\settitle{Generating a pdf for participating medium of non constant density }

Consider some medium where the density is give as
\WfigureNC{density.png}{0.3} %https://www.desmos.com/calculator/ps6mexq4tb
The horizontal is time and the vertical axis is the density of the medium at the position where the ray is at the corresponding time.\\
If the density of the medium was just $\rho_0$, then the pdf would be 
\[f_0(t) = \lambda_0 \e^{-\lambda_0 t} \]
where $\lambda_0$ is a function of $\rho_0$ (as $\rho$ increases, $\lambda$ increases)
(see readme for more explaination).\\ 
Simililarly if the density was just $\rho_1$
\[f_1(t) = \lambda_1 \e^{-\lambda_1 t} \] 
%
Denote the pdf as $f$.\\ 
Then the pdf for $t\in[0,t_0)$ must be the same for a constant medium because the probability to scatter during this interval is uneffected by the presence of the other medium.\\ 
From conditional probability\cite{1}, for $T$ the random variable corresponding to the time of scattering and for $t>t_0$
\[\prob{T>t|T>t_0} = \frac{\prob{T>t, T>t_0}}{\prob{T>t_0}}\] 
but $\set{T>t}$ is a superset of $\set{T>t_0}$ so 
\[\prob{T>t|T>t_0} = \frac{\prob{T>t}}{\prob{T>t_0}}\] 
$\prob{T>t|T>t_0}$ can be interpretted as given that the ray has lasted until $t_0$, what is the probability that is lasts until $t$.
This is equivilent to the ray starting that $t_0$ and lasting a time $t - t_0$.\\ 
As such, 
\[\prob{T>t|T>t_0} = \intt_0^{t-t_0} \lambda_1 \e^{-\lambda_1 t} \dd t = 1 - \e^{\lambda_1 (t-t_0)}\] 
Also, 
\[ \prob{T>t} = 1 - \prob{T<t} = 1 - \intt_0^t f(t) \dd t\] 
Because $T<t_0$ is determined entirely by $f_0$
\[\prob{T>t_0} = 1 - \prob{T<t_0} = 1 - \ab*{1 - \e^{-\lambda_0 t_0}} = \e^{-\lambda_0 t_0}\]
%
Thus,
\[1 - \intt_0^t f(t) \dd t = \ab*{1 - \e^{\lambda_1 (t-t_0)}}\e^{-\lambda_0 t_0} \implies f(t) = \e^{\lambda_1 (t-t_0)}\e^{-\lambda_0 t_0}\] 
and so 
\[ f(t) = \begin{cases}
  \lambda_0 \e^{-\lambda_0 t} & t\in[0,t_0)\\ 
  \e^{\lambda_1 (t-t_0)}\e^{-\lambda_0 t_0} & t\in[t_0,\infty)
\end{cases}\]
%
Similarly for 3 regions of varying density 
\[ f(t) = \begin{cases}
  \lambda_0 \e^{-\lambda_0 t} & t\in[0,t_0)\\ 
  \e^{\lambda_1 (t-t_0)}\e^{-\lambda_0 t_0} & t\in[t_0,t_1)\\
  \e^{\lambda_2 (t-t_1)}\e^{-\lambda_0 t_0}\e^{-\lambda_1(t_1-t_0)} & t\in[t_1,\infty)
\end{cases}\]
and for 4
\[ f(t) = \begin{cases}
  \lambda_0 \e^{-\lambda_0 t} & t\in[0,t_0)\\ 
  \e^{\lambda_1 (t-t_0)}\e^{-\lambda_0 t_0} & t\in[t_0,t_1)\\
  \e^{\lambda_2 (t-t_1)}\e^{-\lambda_0 t_0}\e^{-\lambda_1(t_1-t_0)} & t\in[t_1,t2)\\
  \e^{\lambda_3 (t-t_2)}\e^{-\lambda_0 t_0}\e^{-\lambda_1(t_1-t_0)}\e^{-\lambda_2(t_2-t_1)} & t\in[t_2,\infty)
\end{cases}\]



\newpage
\begin{thebibliography}{9}

  \bibitem{1}
  \url{https://en.wikipedia.org/wiki/Conditional_probability}

\end{thebibliography}

\end{document}
